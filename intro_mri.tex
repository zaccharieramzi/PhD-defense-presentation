\section{Introduction to MRI}

\begin{frame}[plain,c]
    %\frametitle{A first slide}
    
    \begin{center}
        \color{DarkBlue}
    \Huge Magnetic Resonance Imaging~(MRI)
    \end{center}
    
\end{frame}

\begin{frame}{What does an MRI look like?}
    \begin{figure}
        \centering
        \includegraphics[height=0.6\textheight]{Figures/intro_figures/example_knee_fastmri.pdf}
        \caption{\label{fig:mri-example} \textbf{Example of an MR image}: MR image of the knee taken from the fastMRI dataset.\footfullcite{Zbontar}}
    \end{figure}
\end{frame}

\subsection{Importance of MRI}
\begin{frame}{Importance of MRI - 1}
    % how many MRI scans / scanners
    % how likely is it that you will get an MRI in your life
    % Based on a rough extrapolation of the following figure, there is a 99.9\% chance that you will get an MRI in your life in France.
    99.9\% chance you will get an MRI.
    \begin{figure}
        \centering
        \includegraphics[width=\textwidth]{Figures/intro_figures/num_mri_scans.png}
        \caption{\label{fig:num-mri-scans} \textbf{Number of MRI scans per year per 1000 population}: figure courtesy of \citet{OECDMRI}.}
    \end{figure}
\end{frame}

% There is a reason for that: MRI can help diagnose many different conditions
\begin{frame}{Importance of MRI - 2}
    % list of conditions
    % This illustration provides a non-exhaustive list of all the diagnoses that can be carried out with MRI.
    \begin{figure}
        \centering
        \includegraphics[height=0.6\textheight]{Figures/intro_figures/What_can_we_diagnose_with_MRI.pdf}
        \caption{\label{fig:diagnose-mri} \textbf{What can we diagnose with MRI?}  Info compiled from \citet{reimer2010clinical,runge2019essentials}.}
    \end{figure}
\end{frame}

\subsection{Physics of MRI}
% MRI is so popular why isnt it solved already ?
\begin{frame}{Physics of MRI - 1}
    % at its core MRI relies on the MR phenomenon
    % in short: a spin is aligned with the magnetic field, when an RF pulse is sent, it tips the spin in the orthogonal plane before the spin realigns with the magnetic field sending another RF pulse
    \only<1>{
        \vspace{0.5em}

    \begin{figure}
        \centering
        \includegraphics[width=0.14\textwidth]{Figures/intro_figures/Precession_in_magnetic_field.pdf}
        \caption{\label{fig:precession}\textbf{Illustration of the precession of a spin in a magnetic field:} the green arrow represents the $\Bb_0$ magnetic field, while the black arrow represents the magnetic moment of the particle. Illustration courtesy of \citet{wiki}.}
    \end{figure}
    }
    % XXX: maybe change this into a tikzpicture to avoid any weird moving effects and have better arrows and font
    \only<2>{
    \begin{figure}
        \centering
        \includegraphics[width=0.5\textwidth]{Figures/intro_figures/Excitation.pdf}
        \caption{\label{fig:excitation}\textbf{Illustration of the excitation phenomenon:} the blue arrow represents an incoming RF~(Radio Frequency) pulse. Illustration courtesy of \citet{wiki}.}
    \end{figure}
    }
    \only<3>{
    \begin{figure}
        \centering
        \includegraphics[width=0.5\textwidth]{Figures/intro_figures/Relaxation.pdf}
        \caption{\label{fig:relaxation}\textbf{Illustration of the relaxation phenomenon:} the blue arrow represents an outgoing FID~(Free Induction Decay) pulse. Illustration courtesy of \citet{wiki}.}
    \end{figure}
    }
\end{frame}

\begin{frame}{Physics of MRI - 2}
    % Because all spins get excited, we get a global RF pulse that is the weighted sum of the contribution of all spins' relaxation RF pulses: global information
    % We can obtain "multiple global information", by changing a bit the magnetic field spatially using gradients
    % Signal equation
    % However, the FID carries global information.
    FID: global info.
    \pause
    
    % Using magnetic \textbf{gradients} enables changing a bit the magnetic field spatially, and therefore changing the global information depending on local factors.
    Magnetic \textbf{gradients} => change the magnetic field spatially.
    \pause
    
    % This allows us to receive a temporal signal of the form:
    Temporal signal:
    \begin{equation*}
        \vspace{\baselineskip}
        \tikzmarknode{S}{\highlight{blue}{$S_{tr}(t)$}} \propto \omega_0  \int_{V_s} B_{tr} \tikzmarknode{M}{\highlight{green}{$M_{tr}(t, \rb)$}} e^{-\imath \gamma \rb \cdot \int_0^t \tikzmarknode{G}{\highlight{red}{$\Gb(\tau)$}}  \dif \tau} \dif \rb 
    \end{equation*}
    \begin{tikzpicture}[overlay,remember picture,>=stealth,nodes={align=left,inner ysep=1pt},<-]
        % For "S"
        \onslide<4->{
        \path (S.south) ++ (0,-2em) node[anchor=south east,color=blue!67] (exp_S){\textbf{Recorded MR signal}};
        \draw [color=blue!87](S.south) |- ([xshift=-0.3ex,color=blue]exp_S.south west);
        }
        % For "M"
        \onslide<5->{
        \path (M.south) ++ (0, -6em) node[anchor=south east,color=green!77] (exp_M){\textbf{Magnetic field in each location $\rb$,}\\ \textbf{proportional to the spin density $\rhob(\rb)$}};
        \draw [color=green!87](M.south) |- ([xshift=-0.3ex,color=green]exp_M.south west);
        }
        % For "G"
        \onslide<6->{
        \path (G.south) ++ (0, -2em) node[anchor=north west,color=red!47] (exp_G){\textbf{Temporal gradients,}\\\textbf{controlled by the operator}};
        \draw [color=red!87](G.south) |- ([xshift=-0.3ex,color=red]exp_G.south east);
        }
    \end{tikzpicture}
\end{frame}

\begin{frame}{Physics of MRI - 3}
    % Signal equation => k-space
    % The \textbf{k-space} vector, $\kb(t) = \frac{\gamma}{2\pi} \int_0^t \Gb(\tau) \dif \tau$, defines how we traverse the Fourier space of the anatomical image.
    \textbf{k-space} vector: $\kb(t) = \frac{\gamma}{2\pi} \int_0^t \Gb(\tau) \dif \tau$.

    % We are sampling in the Fourier space of the anatomical image
    % XXX maybe review this figure as tikzfigure to have a better looking arrow
    \begin{figure}
        \centering
        \includegraphics[height=0.4\textheight]{Figures/intro_figures/kspace_to_image.pdf}
        \caption{\label{fig:example-kspace}\textbf{Example of a k-space with its corresponding anatomical image}: The raw data is from the fastMRI dataset. The k-space is in log-scale and only the magnitude of the 2 images are represented. We selected only a single coil from the 16 coils available for illustrative purposes.}
    \end{figure}
\end{frame}

\begin{frame}{Physics of MRI - 4}
    % Let's not forget our initial goal here: we want to understand why MRI is slow
    % The relaxation is slow !
    \begin{block}{Recap}
        MRI relies on the nuclear resonance phenomenon. This enables us to sample the Fourier space of the anatomical object of interest.
    \end{block}
    \pause
    MRI is slow, because the \textbf{relaxation} is slow!
\end{frame}

\subsection{Acceleration in MRI}
\begin{frame}{Where is there room for acceleration?}
    % Explain the concept of redundancy
    % first example: partial Fourier => give limits
    % \textbf{Redundancy}, otherwise called \textbf{sparsity, symmetry, structure or a priori information}, is the core concept that will help us accelerate MRI.\\
    \textbf{Redundancy}, or \textbf{sparsity, symmetry, structure or a priori information}, is the key.\\
    
    \begin{overprint}
    \hfill \break
    % Here is an illustrative example:
    An illustrative example:
    \begin{figure}
        \centering
        \onslide<2>\includegraphics[height=0.4\textheight]{Figures/intro_figures/astronaut_masked.pdf}
        \onslide<3>\includegraphics[height=0.4\textheight]{Figures/intro_figures/astronaut.pdf}
        \caption{\label{fig:astronaut-masked}\textbf{The inpainting problem.} Even without access to all the pixel values directly, we can infer them, because the information in the image is redundant.
        }
    \end{figure}
    
    \onslide<4>
        \hfill \break
        Is there a similar thing in MRI?\\

        % Yes! The anatomical image is real-valued so its Fourier Transform~(FT) has a conjugate symmetry.
        Yes! Fourier Transform~(FT) has a conjugate symmetry => \textbf{Partial Fourier}.\\
        % Using this redundancy to sample less points in the k-space (i.e. using the relaxation fewer times) is a technique called \textbf{Partial Fourier}.
        
        % But in practice it is still needed to sample 6/8 of the Fourier space (acceleration of 1.3).\footfullcite{emri}
        Resulting acceleration: 1.3.
    
    \end{overprint}
    
\end{frame}

\begin{frame}{Parallel imaging}
    % "forging" the redundancy
    % We can build more redundancy in the measuring system by using \textbf{more antennas (called coils)} to measure the magnetic signal.
    Build more redundancy: \textbf{more antennas (called coils)} => \textbf{Parallel Imaging~(PI)}.\\
    
    % This technique is called \textbf{Parallel Imaging~(PI)}. 
    % A reconstruction algorithm is now needed to handle the multi-coil undersampled data.
        % \textbf{SENSE}\footfullcite{Pruessmann1999SENSE:MRI} and \textbf{GRAPPA}\footfullcite{Griswold2002GeneralizedGRAPPA} are such algorithms.    

    Multi-coil reconstruction algorithms: \textbf{SENSE}\footfullcite{Pruessmann1999SENSE:MRI} and \textbf{GRAPPA}\footfullcite{Griswold2002GeneralizedGRAPPA}.
    
\end{frame}

\begin{frame}{The example of GRAPPA}
    \begin{figure}
        \centering
        \includegraphics[height=0.6\textheight]{Figures/intro_figures/GRAPPA.jpeg}
        \caption{\label{fig:GRAPPA}\textbf{GRAPPA illustration.} Image courtesy of \citet{deshmane2012parallel}.
        }
    \end{figure} 
\end{frame}

\begin{frame}{Limits of Parallel Imaging}
    % Max AF
    % XXX: wait for AV's response
    % ow: https://www.siemens-healthineers.com/magnetic-resonance-imaging/options-and-upgrades/clinical-applications/syngo-grappa says 2,3
\end{frame}