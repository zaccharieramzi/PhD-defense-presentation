\section{Deep Learning}

\subsection{The power of Deep Learning}
\begin{frame}{The power of Deep Learning}
    % we want to learn a complicated function that tells us whether an image is an MR image
    % similarly deep learning has been able to build functions that tell whether an image is that of a dog or a cat
    % universal approx
    We want to learn a complicated function that tells us whether a complex-valued vector is an MR image or not.
    \pause

    Deep Learning~(DL) has been used to build such functions:
    % XXX: image that does f_theta(Dog Image) -> dog
\end{frame}

\begin{frame}{Formalism - 1}
    % supervised learning objective function
    The classical framework for DL is supervised learning:
    \vspace{\baselineskip}

    \begin{equation*}
        \argmin_{\tikzmarknode{params}{\highlight{orange}{$\thetab \in \Theta$}}} \tikzmarknode{sum}{\highlight{purple}{$\sum\limits_{(\xb_i, \yb_i) \in \mathcal{D}}$}} \tikzmarknode{loss}{\highlight{green}{$\mathcal{L}$}}(\tikzmarknode{nn}{\highlight{brown}{$f_{\thetab}$}} (\tikzmarknode{input}{\highlight{blue}{$\xb_i$}}), \tikzmarknode{label}{\highlight{red}{$\yb_i$}}, \thetab)
    \end{equation*}
    \begin{tikzpicture}[overlay,remember picture,>=stealth,nodes={align=left,inner ysep=1pt},<-]
        % For "input"
        \onslide<2->{
        \path (input.north) ++ (0,1.5em) node[anchor=south west,color=blue!87] (exp_input){
            input};
        \draw [color=blue!87](input.north) |- ([xshift=-0.3ex,color=blue]exp_input.south east);
        }
        % For "label"
        \onslide<3->{
        \path (label.south) ++ (0, -2.5em) node[anchor=south west,color=red!87] (exp_label){
            label};
        \draw [color=red!87](label.south) |- ([xshift=-0.3ex,color=red]exp_label.south east);
        }
        % For "nn"
        \onslide<4->{
        \path (nn.south) ++ (0, -4.5em) node[anchor=south east,color=brown!87] (exp_nn){
            neural network};
        \draw [color=brown!87](nn.south) |- ([xshift=-0.3ex,color=brown]exp_nn.south west);
        }
        % For "loss"
        \onslide<5->{
        \path (loss.north) ++ (0, 1.5em) node[anchor=south west,color=green!47] (exp_loss){
            loss};
        \draw [color=green!47](loss.north) |- ([xshift=-0.3ex,color=green]exp_loss.south east);
        }
        % For "sum"
        \onslide<6->{
        \path (sum.south) ++ (0, -2em) node[anchor=south east,color=purple!87] (exp_sum){
            Estimator of the expected value};
        \draw [color=purple!87](sum.south) |- ([xshift=-0.3ex,color=purple]exp_sum.south west);
        }
        
        % For "params"
        \onslide<7->{
        \path (params.west) ++ (-2em, 0) node[anchor=east,color=orange!87] (exp_params){
            Parameters};
        \draw [color=orange!87](params.west) -- ([xshift=-0.3ex,color=orange]exp_params.east) -- ([xshift=-0.3ex,color=orange]exp_params.south east) -- ([xshift=-0.3ex,color=orange]exp_params.south west);
        }
        
    \end{tikzpicture}
\end{frame}

\begin{frame}{Formalism - 2}
    % Stochastic Gradient descent and chain rule
    To solve the previous equation we will use two main tools:
    
    \begin{enumerate}
        \item \alt<2>{Stochastic Gradient Descent~(SGD)}{\highlight{blue}{Stochastic Gradient Descent~(SGD)}};
        \item<2> \highlight{blue}{Chain rule}.
    \end{enumerate}

    
        \begin{block}{Definition}
            \only<1>{An algorithm to solve the previous optimization problem based on first order derivatives.}
            \only<2>{A property allowing us to compute easily derivatives of compound functions.}
        \end{block}    
    
\end{frame}

\subsection{Requirements for Deep Learning}
\begin{frame}{Requirements for Deep Learning}
    % Great that I can do that, but does it take ?
    % Data, compute + memory, framework
    % accept that it's "black-box"
    What does it take to use DL in a problem?
    % XXX remove semi colon in itemize over the whole pres
    \begin{itemize}[<+->]
        \item data;
        \item compute \& memory;
        \item development framework;
        \item accepting that it's "black-box".
    \end{itemize}
\end{frame}

% \begin{frame}{Building the network}
%     % give classical functions
% \end{frame}

\begin{frame}{Introduction Recap}
    \begin{block}{Recap}
        MRI is slow because of relaxation.
        
        \pause
        If we want to do fewer relaxations, we need to exploit some redundancy in MR images.
        
        \pause
        But this redundancy is not easy to express with handcrafted linear functions.
        
        \pause
        This is why we want to use Deep Learning which enables the calibration of complicated function.
    \end{block}
\end{frame}
