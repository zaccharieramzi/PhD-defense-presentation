%----------------------------------------------------------------------------------------
%	PACKAGES AND THEMES
%----------------------------------------------------------------------------------------
\documentclass[aspectratio=169,xcolor=dvipsnames]{beamer}
\usetheme{SimplePlus}

\usepackage{hyperref}
\usepackage{graphicx} % Allows including images
\usepackage{booktabs} % Allows the use of \toprule, \midrule and \bottomrule in tables

%----------------------------------------------------------------------------------------
%	TITLE PAGE
%----------------------------------------------------------------------------------------

\title[Advanced deep neural networks for MRI image reconstruction]{Advanced deep neural networks for MRI image reconstruction from highly undersampled data in challenging acquisition settings
} % The short title appears at the bottom of every slide, the full title is only on the title page
\subtitle{PhD defense}

\author[Zaccharie] {Zaccharie Ramzi}

\institute[Inria-CEA] % Your institution as it will appear on the bottom of every slide, may be shorthand to save space
{
    Parietal team, Inria Saclay \\
    NeuroSpin and Cosmostat, CEA Saclay
}
\date{18th February 2022} % Date, can be changed to a custom date


%----------------------------------------------------------------------------------------
%	PRESENTATION SLIDES
%----------------------------------------------------------------------------------------

\begin{document}

\begin{frame}
    % Print the title page as the first slide
    \titlepage
\end{frame}

% Introduction: the problem at hand
% Introduction: physics of MRI
% Acceleration: classics
% Acceleration: CS
% Deep Learning
% Application of DL to MRI: model agnostic, single domain and unrolled networks
% Clinical applicability
% Going further !
% CCL

\begin{frame}
    % I would like to have the picture of inside the MRI
    % and pass the sound of an MRI scan
\end{frame}

% link between the 2: this is the sound you hear when undergoing an MRI
% now imagine that you will on average hear this during x minutes
% MRI scanners being slow not only generate discomfort but have other impacts on their efficiency...
\begin{frame}{MRI is slow}
    % give typical duration (compare to CT)
    % give associated problems
\end{frame}

\begin{frame}{Our objective: accelerate MRI scans}
    % give TOC
\end{frame}

\section{Introduction to MRI}
\subsection{Importance of MRI}
\begin{frame}{Importance of MRI - 1}
    % how many MRI scans / scanners
    % how likely is it that you will get an MRI in your life
\end{frame}

% There is a reason for that: MRI can help diagnose many different conditions
\begin{frame}{Importance of MRI - 2}
    % list of conditions
\end{frame}

\subsection{Physics of MRI}
% MRI is so popular why isnt it solved already ?
\begin{frame}{Physics of MRI - 1}
    % at its core MRI relies on the MR phenomenon
    % in short: a spin is aligned with the magnetic field, when an RF pulse is sent, it tips the spin in the orthogonal plane before the spin realigns with the magnetic field sending another RF pulse
    % video of e-MRI
\end{frame}

\begin{frame}{Physics of MRI - 2}
    % Because all spins get excited, we get a global RF pulse that is the weighted sum of the contribution of all spins' relaxation RF pulses: global information
    % We can obtain "multiple global information", by changing a bit the magnetic field spatially using gradients
    % Signal equation
\end{frame}

\begin{frame}{Physics of MRI - 3}
    % Signal equation => k-space
    % We are sampling in the Fourier space of the anatomical image
\end{frame}

\begin{frame}{Physics of MRI - 4}
    % Let's not forget our initial goal here: we want to understand why MRI is slow
    % The relaxation is slow !
\end{frame}

\subsection{Acceleration in MRI}
\begin{frame}{Where is there room for acceleration?}
    % Explain the concept of redundancy
    % first example: partial Fourier => give limits
\end{frame}

\begin{frame}{Parallel imaging}
    % "forging" the redundancy
    % GRAPPA + SENSE examples
\end{frame}

\section{Compressed Sensing}
\begin{frame}{Limits of Parallel Imaging}
    % Max AF
\end{frame}

%----------------------------------------------------------------------------------------

\end{document}