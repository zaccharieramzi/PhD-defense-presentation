\usepackage[dvipsnames]{xcolor}
\usepackage{tikz}
\usetikzlibrary{backgrounds}
\usetikzlibrary{arrows,shapes}
\usetikzlibrary{tikzmark}
\usetikzlibrary{calc}

\usepackage{mathtools, nccmath}
\usepackage{wrapfig}
\usepackage{comment}



% table alignment
\usepackage{array}
\usepackage{ragged2e}
\newcolumntype{P}[1]{>{\RaggedRight\hspace{0pt}}p{#1}}
\newcolumntype{X}[1]{>{\RaggedRight\hspace*{0pt}}p{#1}}

% color box
\usepackage{tcolorbox}


% for tikz
\usepackage{tikz}
%\usetikzlibrary{trees}
\usetikzlibrary{arrows,shapes,positioning,shadows,trees,mindmap}
% \usepackage{forest}
\usepackage[edges]{forest}
\usetikzlibrary{arrows.meta}
\colorlet{linecol}{black!75}
\usepackage{xkcdcolors} % xkcd colors


% for colorful equation
\usepackage{tikz}
\usetikzlibrary{backgrounds}
\usetikzlibrary{arrows,shapes}
\usetikzlibrary{tikzmark}
\usetikzlibrary{calc}
% Commands for Highlighting text -- non tikz method
\newcommand{\highlight}[2]{\colorbox{#1!17}{$\displaystyle #2$}}
%\newcommand{\highlight}[2]{\colorbox{#1!17}{$#2$}}
\newcommand{\highlightdark}[2]{\colorbox{#1!47}{$\displaystyle #2$}}

% my custom colors for shading
\colorlet{mhpurple}{Plum!80}


% Commands for Highlighting text -- non tikz method
\renewcommand{\highlight}[2]{\colorbox{#1!17}{#2}}
\renewcommand{\highlightdark}[2]{\colorbox{#1!47}{#2}}

% Some math definitions
\newcommand{\lap}{\mathrm{Lap}}
\newcommand{\pr}{\mathrm{Pr}}

\newcommand{\Tset}{\mathcal{T}}
\newcommand{\Dset}{\mathcal{D}}
\newcommand{\Rbound}{\widetilde{\mathcal{R}}}